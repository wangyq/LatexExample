%% example of algorithem and source code
%%
\documentclass{article}

    \usepackage{epsfig}
    
    %\usepackage{algorithm2e}
    %\usepackage[linesnumbered]{algorithm2e}
    %\usepackage[linesnumbered,ruled]{algorithm2e}
    \usepackage[linesnumbered,ruled,vlined]{algorithm2e}
    

    %% for mathematics use
    \usepackage{amssymb,amsmath}

    %% for Source Code use
    % sourcecode for color
  \usepackage{xcolor,inconsolata}
  \newcommand{\Cpp}{C\texttt{++}}

  %% listing for source code
  \usepackage{listings}

  \lstset{
  language = C++,
  basicstyle = \ttfamily,
  keywordstyle = \color{blue}\textbf,
  commentstyle = \color{gray},
  stringstyle = \color{green!70!black},
  stringstyle = \color{red},
  columns = fullflexible,
  numbers = left,
  numberstyle = \scriptsize\sffamily\color{gray},
  caption = A hello world program in \Cpp,
  xleftmargin = 0.16\textwidth,
  xrightmargin = 0.16\textwidth,
  showstringspaces = false,
  float,
  }


    %% title and author
    \title{Example of how to use Algorithm2e}
    \author{Robert Woodward}
    
    \begin{document}
    \maketitle
    
    \section{Mathematics Example}

    From $ a+b>c $, we have $X = x^2$...

    %% unnumbered section
    \subsection*{The golden ratio}
    
    The symbol for the golden ratio is the Greek letter \( \varphi \). Its value is the positive solution of \( x^2 - x - 1 = 0 \).

It can be calculated to:
\[
\varphi = \frac{1 + \sqrt{5}}{2} = 1.618 \ldots
\]

    \section{Source Code Example}

    Following is the C++ source code example of "Hello, world!"

  %\begin{lstlisting}[language = C++]
  \begin{lstlisting}
      // include standard input/output stream objects:
      #include <iostream>
      // the main method:
      int main() {
      std::cout << "Hello TeX world!" << std::endl;
      }
  \end{lstlisting}


  \section{Algorithm Example}

  % Algorithm 1: While loop example with If/Else condition
 
 
\begin{algorithm}[H]
  \DontPrintSemicolon
  \SetAlgoLined
  \KwResult{Write here the result}
  \SetKwInOut{Input}{Input}\SetKwInOut{Output}{Output}
  \Input{Write here the input}
  \Output{Write here the output}
  \BlankLine
   
  \While{While condition}{
      instructions\;
      \eIf{condition}{
          instructions1\;
          instructions2\;
      }{
          instructions3\;
      }
  }
   
  \caption{While loop with If/Else condition}
  \end{algorithm}


  \bigskip % add 12pt space in-between
 
% Algorithm 2: Nested ForEach loop example with If/ElseIf/Else condition
 
 
\begin{algorithm}[H]
\DontPrintSemicolon
\SetAlgoLined
\KwResult{Write here the result}
\SetKwInOut{Input}{Input}\SetKwInOut{Output}{Output}
\Input{Write here the input}
\Output{Write here the output}
\BlankLine
 
$x\leftarrow 0$    \;
$y\leftarrow 0$    \;
\BlankLine
\ForEach{ForEach condition}{    
    
    \BlankLine
    
    \tcc{comments on code}    
    \ForEach{ForEach condition}{
        \If{If condition}{
            instruction(s) like below: \\
            increase $x$ by $1$\;
            decrease $y$ by $2$\;
        }
                
        \BlankLine
        
        \uIf{If condition}{
            instruction
        }
        \uElseIf{ElseIf condition}{
            instruction
        }
        \uElse{
            instruction
        }                
    }    
}
\caption{Nested ForEach loop with If/ElseIf/Else condition}
\end{algorithm}


  \section{Algorithem2e Example by Robert Woodward}

    Below we illustrate the formatting as pseudo code of some sample
    of simple algorithms. The goal is not to entice you to use
    \LaTeX\/ for formatting your algorithms as currently the best
    possible formatting tool for algorithms. Please carefully check
    the source files and learn how to use this style. Importantly:
    \begin{itemize}
    \item
    Always state your input
    \item
    State the output if any
    \item
    Always number your lines for quick referral.
    \item
    Always declare and initialize your local variables
    \item
    Always use \textbackslash gets for assignments
    \item
    Always end with ``return'' even when not returning any values
    \item
    Use common functions and operands such as {\sc Union}, {\sc PowerSet}, etc. as often as needed, unless you are asked to define them.
    \end{itemize}
    
    Algorithm~ \ref{algo:max} will find the maximum element in a finite sequence (Slide 14 in Class Slides).
    
    \begin{algorithm}
    \DontPrintSemicolon % Some LaTeX compilers require you to use \dontprintsemicolon instead
    \KwIn{A finite set $A=\{a_1, a_2, \ldots, a_n\}$ of integers}
    \KwOut{The largest element in the set}
    $max \gets a_1$\;
    \For{$i \gets 2$ \textbf{to} $n$} {
      \If{$a_i > max$} {
        $max \gets a_i$\;
      }
    }
    \Return{$max$}\;
    \caption{{\sc Max} finds the maximum number}
    \label{algo:max}
    \end{algorithm}
    
    Algorithm~\ref{algo:change} is a greedy change-making algorithm (Slide 19 in Class Slides).
    
    \begin{algorithm}
    \DontPrintSemicolon % Some LaTeX compilers require you to use \dontprintsemicolon instead 
    \KwIn{A set $C = \{c_1, c_2, \ldots, c_r\}$ of denominations of coins, where $c_i > c_2 > \ldots > c_r$ and a positive number $n$}
    \KwOut{A list of coins $d_1,d_2,\ldots,d_k$, such that $\sum_{i=1}^k d_i = n$ and $k$ is minimized}
    $C \gets \emptyset$\;
    \For{$i \gets 1$ \textbf{to} $r$}{
      \While{$n \geq c_i$} {
        $C \gets C \cup \{c_i\}$\;
        $n \gets n - c_i$\;
      }
    }
    \Return{$C$}\;
    \caption{{\sc Change} Makes change using the smallest number of coins}
    \label{algo:change}
    \end{algorithm}
    
    Algorithm~\ref{algo:duplicate} and Algorithm~\ref{algo:duplicate2} will find the first duplicate element in a sequence of integers.
    
    \begin{algorithm}
    \DontPrintSemicolon % Some LaTeX compilers require you to use \dontprintsemicolon instead
    \KwIn{A sequence of integers $\langle a_1, a_2, \ldots, a_n \rangle$}
    \KwOut{The index of first location witht he same value as in a previous location in the sequence}
    $location \gets 0$\;
    $i \gets 2$\;
    \While{$i \leq n$ \textbf{and} $location = 0$}{
      $j \gets 1$\;
      \While{$j < i$ \textbf{and} $location = 0$}{
        % The "u" before the "If" makes it so there is no "end" after the statement, so the else will then follow
        \uIf{$a_i = a_j$}{
          $location \gets i$\;
        }
        \Else{
          $j \gets j + 1$\;
        }
      }
      $i \gets i + 1$\;
    }
    \Return{location}\;
    \caption{{\sc FindDuplicate}}
    \label{algo:duplicate}
    \end{algorithm}
    
    \begin{algorithm}
    \DontPrintSemicolon
    \KwIn{A sequence of integers $\langle a_1, a_2, \ldots, a_n \rangle$}
    \KwOut{The index of first location witht he same value as in a previous location in the sequence}
    $location \gets 0$\;
    $i \gets 2$\;
    \While{$i \leq n \land location = 0$}{
      $j \gets 1$\;
      \While{$j < i \land location = 0$}{
        % The "l" before the If makes it so it does not expand to a second line
        \lIf{$a_i = a_j$}{
          $location \gets i$\;
        }
        \lElse{
          $j \gets j + 1$\;
        }
      }
      $i \gets i + 1$\;
    }
    \Return{location}\;
    \caption{{\sc FindDuplicate2}}
    \label{algo:duplicate2}
    \end{algorithm}
    
    \end{document}